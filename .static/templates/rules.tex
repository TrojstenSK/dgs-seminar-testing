\pagestyle{rules}

\section{Pravidlá a postihy}
    \begin{itemize}
        \item FKS môžeš riešiť, ak si študentom strednej školy, gymnázia,
            prípadne základnej školy na Slovensku alebo v zahraničí.
        \item Počas školského roka prebehnú dve samostatné (nezávislé) časti súťaže (semestre):
            zimná a letná. Každá časť sa skladá z troch kôl úloh. Každé kolo obsahuje osem úloh.
            Na vyriešenie každého kola úloh budeš mať približne tri týždne času.
        \item Za každý príklad môžeš dosťať 0 až 9 bodov, ktoré sa ti počas celého semestra sčítavajú,
            a na základe ktorých sa určí po každom semestri výsledné poradie riešiteľov.
            Najlepších riešiteľov pozývame na týždňové sústredenia (zimné a letné).
        \item Keďže FKS je súťaž jednotlivcov, v prípade opisovania sme nekompromisní a strhávame zaň body.
        \item Začínajúci riešitelia môžu riešiť všetkých 8 úloh, skúsenejší riešitelia postupne strácajú
            možnosť súťažne riešiť najľahšie úlohy. Do výsledkov súťaže sa každému započítavajú \textbf{štyri}
            najlepšie vyriešené úlohy z~každého kola. To, kto môže riešiť ktoré úlohy,
            je určené systémom levelov. Dosiahnuteľné levely sú 1 až 4,
            riešiteľ s levelom $L$ môže riešiť úlohy s číslami $L$ až 8.
    \end{itemize}

\subsection{Ako má vyzerať moje riešenie?}
    \begin{itemize}
       \item Ako v~mnohých iných súťažiach, aj tu platí jednoduchá zásada -- písať všetko, čo
            o~príklade vieš. Teda, aj keď nevieš celé riešenie, oplatí sa spísať aspoň časti
            riešenia (názory, postrehy, pokusy, náčrty). Pokiaľ však o~svojom riešení vieš, že
            nie je úplné, určite to napíš!

            Neprepadaj panike! Ak príklad nevieš vyriešiť, pravdepodobne to znamená, že je
            ťažký. Ak je ťažký pre teba, tak je zrejme ťažký aj pre iných. Nikto nevraví,
            že musíš byť v~prvej trojke. Aj 12. miesto je úspech -- minimálne z~hľadiska
            pozvania na sústredko.

            Ak nepochopíš úplne presne zadanie príkladu, môžeš sa nás na podrobnosti opýtať
            e-mailom na \URL{(* competition.email *)}.
        \item Riešenia akceptujeme iba v elektronickej podobe a len vo formáte \texttt{pdf}.
            Ak však stále uprednostňuješ písanie riešenia na papier,
            môžeš svoje riešenie oskenovať a uložiť každú takto naskenovanú úlohu
            do jedného samostatného \texttt{pdf} súboru.

    \end{itemize}

\subsection{Chcem začať riešiť! Čo mám spraviť?}
    Pred riešením sa musíš zaregistrovať na stránke \URL{(* competition.URL *)}
    a vyplniť požadované kontaktné údaje, ak si tak už neurobil pri registrácii do iného seminára Trojstenu.
    Odporúčame sa zaregistrovať pár dní pred odovzdávaním tvojho riešenia
    (pre prípad, že by si mal počas registrácie nejaké problémy).

\subsection{Prijatie na FMFI UK bez prijímačiek}
    Ak účastník získa v niektorej časti (zimnej, letnej) a ľubovoľnej kategórii
    FKS aspoň 65\% celkového počtu bodov a hlási sa na študijný program,
    ktorého profilovým predmetov je fyzika, bude prijatý.

    Ba čo viac, ak dosiahne excelentné výsledky a dostane za to Dekanský list.
    V prípade, že príde študovať na FMFI UK, čaká naňho motivačné štipendium vo výške približne 300 eur.

\hfill \emph{Veľa zdaru Ti prajú Tvoji vedúci!}%
